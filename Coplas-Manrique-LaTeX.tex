\documentclass[11pt,a4paper,twoside]{article}
\usepackage[noeledsec,noend,noledgroup,nopenalties,series={A,E},parapparatus]{reledmac}
\usepackage{fontspec}
\usepackage[paperwidth=17.00cm,paperheight=25.00cm,top=1.40cm,inner=1.95cm,outer=1.65cm,bottom=2.50cm]{geometry}
\usepackage{polyglossia}
\usepackage[breaklinks=true,hidelinks]{hyperref}
\usepackage{tikz}
\usepackage[backend=biber,style=apa,url=true]{biblatex}
\addbibresource{bibliography.bib}
%
\title{\huge \textbf{COPLAS SOBRE LA MVERTE DE SV PADRE}}
\author{\huge\textsc{Jorge Manriqve}}
\date{\vspace{60pt} \textsc{Edición crítica liminar}\\\fontsize{40}{11}\selectfont{\vfill\LaTeX}}
%
\setmainfont{TemporaNR}[RawFeature=onum]
\setdefaultlanguage{spanish}
\setotherlanguage[variant=ancient]{greek}
%
\setlength{\parindent}{0.6cm}
\setlength{\bibhang}{0.6cm}
%
\usetikzlibrary{trees}
%
\fnpos{critical-familiar}
\fnpos{{A}{familiar},{A}{critical},{E}{familiar}}
\Xarrangement{paragraph}
\arrangementX[A]{paragraph}
\AtBeginDocument{\Xmaxhnotes{0.66\textheight}\maxhnotesX{1\textheight}}
\Xinplaceoflemmaseparator{0em}
\Xnumberonlyfirstinline
\Xnumberonlyfirstintwolines
\parindentX[E]
\Xnotenumfont{\bfseries}
\Xsymlinenum{\char"02016}
\Xbeforesymlinenum{-0.6em}
\Xafterlemmaseparator{0.4em}
\Xbeforelemmaseparator{-0.1em}
\Xafternumber{0.4em}
\linenummargin{outer}
%
\prenotesX{11pt}
\afterruleX{2pt}
\Xprenotes{11pt}
\Xafterrule{2pt}
%
\newcommand{\comillas}[1]{«#1»}
\renewcommand{\thefootnoteA}{}
%
\usepackage{tocloft}
\renewcommand{\cftsecleader}{\cftdotfill{\cftdotsep}}
\renewcommand{\cfttoctitlefont}{\bfseries\scshape}
%
%Para que se cite correctamente el shortauthor.  
\makeatletter
\AtEveryCitekey{\ifnameundef{shortauthor}{}{\def\cbx@apa@ifnamesaved{\@firstoftwo}}}
\makeatother
%Espaciado de la lista.
\usepackage{enumitem}
\setlist{nosep,noitemsep,parsep=0pt}
%
%CONSPECTVS SIGLORVM
\newcommand{\eg}{{\emph{LB3}}}
\newcommand{\jm}{\emph{82*JM}}
\newcommand{\rl}{\emph{86*RL}}
\newcommand{\hh}{\emph{HH1}}
\newcommand{\mn}{\emph{MN19}}
\newcommand{\im}{\emph{83*IM}}
%
%
\begin{document}
	{\pagenumbering{gobble}
	\maketitle
}
	\newpage
%
\hskip0pt
\vfill
\begin{flushright}
Edición y composición en LuaLaTeX con el paquete\\ Reledmac de \textsc{Maïaeul Rouquette} por \textsc{Ioannes Bracledius}.
\end{flushright}
\newpage
%
\tableofcontents
\newpage
%
\pagenumbering{arabic}
\setcounter{page}{1}
\section*{\centering \textsc{Prefacio}}
\addcontentsline{toc}{section}{\textsc{Prefacio}}

\textsc{Jorge Manrique nació} ca. 1440 y murió en abril de 1479. Una noticia de su muerte la transmitió Fernando del Pulgar en su \emph{Crónica de los reyes Católicos}:\par
%
\comillas{... peleaban los más días con el marqués de Villena e con su gente, e había entre ellos algunos recuentros. En uno de los quales, el capitán don Jorge Manrique se metió con tanta osadía entres los enemigos que, por no ser visto de los suyos para que fuera socorrido, le firieron de muchos colpes, e murió peleando cerca de las puertas del castillo de Garcimuñoz, donde acaeció aquella pelea, en la qual murieron algunos escuderos e peones de la una e la otra parte.}\footnoteE{[De la guerra que se fizo contra el marqués de Villena en Escalona y en el Marquesado]. En la ed. de Carriazo, aquí utilizada, vol. I, se ubica en la pág. 358; en la \emph{editio princeps}, atribuida a Nebrija, de 1565, en el cap. C, f. 120r; en la ed. de 1780 (Valencia), en \comillas{Segunda parte}, cap. LXXXII, pág. 146.}\par
%
Las \emph{Coplas sobre la muerte de su padre} son una \comillas{colección} de coplas compuestas y dedicadas a don Rodrigo Manrique, padre del autor y maestre de la Orden de Santiago, hacia fines de 1476, tras su muerte. El texto presenta una figura ensalzada del maestre como victorioso guerrero que \comillas{fizo guerras a moros} [\emph{bellator}], hombre virtuoso y católico honrado. Utiliza la \comillas{copla de pie quebrado} o \comillas{sextina manriqueña} --cuatro octosílabos y dos tetrasílabos con rimas consonantes--, lenguaje llano y culto, y posee erudición histórica, filosófica y teológica.\par
%
El título, \emph{Coplas sobre la muerte de su padre}, resulta un tanto arbitrario si se analizan las rúbricas que la tradición textual transmitió: los testimonios transmitieron \comillas{dezir} (\jm), \comillas{coplas} (\eg, \hh, \im) y algún otro \comillas{De don jorge manrique} (\rl). Parece evidente, pues, que el título no es reconstruible, porque todos aquellos indican el \comillas{tema} o \comillas{motivación} y, además, hace referencia a la estructura formal de la composición; y aunque por mayoría se prefiere \comillas{coplas}, no resulta sensato sostener que aquel, en sus variantes, es el título propiamente de la \comillas{colección}. Es probable que, como algunos textos medievales, este no se haya concebido con un título como tal.\par
%
No se conservaron autógrafos de alguna copla ni testimonios que puedan considerarse apógrafos o idiógrafos. El testimonio más antiguo que contiene la colección de cuarenta coplas es el manuscrito del \emph{Cancionero de Egerton}, y no es sino hasta ca. 1482-83 --seis o siete años tras la redacción y cuatro o cinco tras el fallecimiento del autor-- que aparece en un impreso en carácter de \emph{editio princeps}. Es necesario destacar que los textos de Manrique solo se conservaron disueltos en cancioneros, a diferencia de otro escritor como Luis de Góngora, que habría gozado de \comillas{colecciones} con su obra.\par

\section*{\centering \textsc{Testimonios}}
\addcontentsline{toc}{subsection}{\textsc{Testimonios}}
%
Para las siglas de los testimonios se adoptaron las utilizadas por \textcite{PérezPriego2017} quien, a su vez, 
las adoptó de Dutton y su trabajo \emph{El cancionero del s. XV (ca. 1360-1520)}. Los testimonios son:\par
%
{\eg} = \emph{Cancionero de Egerton} (British Library, Eg. 939).
%
Manuscrito en papel con marcas de agua de fines de los 1470 o inicios de los 1480; 19x14,5cm.; escritura gótica cortesana, copista castellano; copia de un genuino llevado a cabo en la corte de Navarra; contiene textos poéticos y epistolares; 3 guardas [2 al inicio y 1 de cierre] y 122 fols.; los ff. 15r--18v contienen las coplas con la rúbrica \comillas{¶Coplas que fiço don jorje manrriq\emph{ue} sobre la muerte del maestre de santiago don rrodrigo manrriq\emph{ue} su padre}; la copla XL estuvo incompleta, pero otra redacción la completó en los pasajes restantes con una escritura gótica descuidada, escrita en los espacios restantes del f. 18v, porque el fol. central [primitivo f. 19v] no se conservó: tras aquel debía continuar el texto del \emph{Infante Epitus} \textcite{Beltrán2011}; \comillas{cancionero pro-Isabelino} \parencite{Severin2000}. <\url{https://www.cervantesvirtual.com/nd/ark:/59851/bmcj1034}>\par

{\jm} = \emph{Vita Christi por fray Íñigo de Mendoza} (RBME, 100-II-17).
%
Impresión incunable; tipografía gótica a doble columna; ca. 1480-1484, Zaragosa?; 52 fols.; los ff. 41r--44v contienen las coplas con la rúbrica \comillas{Dezir de don jorge manrriq\emph{ue} por la muerte de su padre}; hay dos ejemplares y se hallan: uno, en la Real Biblioteca del Monasterio de San Lorenzo de El Escorial [100-II-17], y el otro, en la Biblioteca Comunale de Palermo [Esp-XI, F. 56, núm. 3]; plausible \emph{editio princeps}. <\url{https://rbme.patrimonionacional.es/s/rbme/item/14443#?xywh=-1601%2C-39%2C4854%2C2165}>\par

{\rl} = \emph{Cancionero de Ramón de Llavia} (BNE, INC/2567).
%
Impresión a doble columna; tipografía gótica; Zaragoza, Pablo Hurus, ca. 1488-1490?; 98 fols.; los ff. 73r--76r contienen las coplas con la rúbrica \comillas{De don jorge manrrique por la muerte de su padre}; dos ejemplares en la Biblioteca Nacional de España con las siglas INC/2567 y INC/2892; dedicado a doña Francisquina de Bardací, según el prólogo \comillas{fecho a la señora doña francisqujna de bardaçi mujer del magnifico señor mofsen(?) joan fernandez de heredia gouernador de arago\emph{n} por ramon dellabia en q\emph{ue} le endereça el presente libro}. <\url{https://bdh-rd.bne.es/viewer.vm?id=0000108057&page=1}>\par

{\hh} = \emph{Cancionero de Oñate-Castañeda} (Harvard Houghton Library, MS. Span. 97).
%
Manuscrito en papel; 28x20,6cm.; escritura gótica del s. XV con rasgos de cortesana; papel y filigranas ca. 1480-1486; consta de un total de 437 fols.; los ff. 421v--424v contienen las coplas I-XXXIV con la rúbrica \comillas{Coplas de do\emph{n} jorge ma\emph{n}riq\emph{ue} a la muerte del maestre do\emph{n} rod\emph{rig}o ma\emph{n}riq\emph{ue} su padre}; carece del f. 425; para las cuestión de las variantes adiáforas \emph{conservados}/\emph{olvidados} resulta lacunoso. Edición paleográfica de \textcite{Severin1990}.\par

{\mn} = \emph{Cancionero de Pero Guillén de Segovia} (BNE, MS/4114).
%
Manuscrito en papel; 21x15cm.; copia del s. XVIII de un manuscrito de fines del s. XV (Pérez Priego, 2017; Beltrán, 1991); 6 guardas [3 al inicio y 3 de cierre] y 731 fols.; los ff. 407r--418v contienen las coplas X-XL [acéfalo]. <\url{https://bdh-rd.bne.es/viewer.vm?id=0000134993&page=1}>\par

{\im} = \emph{Cancionero de fray Íñigo de Mendoza} (BNE, INC/897).
%
Impresión incunable a doble columna; tipografía gótica; Zamora, 1483?; 90 fols.; los ff. 67r--70r contienen las coplas con la rúbrica \comillas{Coplas que hizo do\emph{n} jorge manrriq\emph{ue} ala muerte del maestre de santiago do\emph{n} rodrigo manrique su padre}; existen dos ejemplares y se hallan uno, en la Biblioteca Nacional de España [INC/897], y otro, en la Real Biblioteca del Monasterio de San Lorenzo de El Escorial [38-I-22]. <\url{https://bdh-rd.bne.es/viewer.vm?id=0000176297&page=1}>\par
\relax
\vfill

\section*{\centering \textsc{Criterios de edición}}
\addcontentsline{toc}{subsection}{\textsc{Criterios de edición}}

Para la fijación del texto se tuvo en cuenta el trabajo filológico de \textcite{Beltrán1991,Beltrán2013} y \textcite{PérezPriego1990,PérezPriego2017} y, conforme a ello, se siguió el subarquetipo \emph{α} del \emph{stemma codicum} construido por el segundo. Descartando el subarquetipo \emph{β}, siendo {\im} su testimonio más representativo\footnoteE{Aunque el subarquetipo \emph{β} contenga otros testimonios útiles para la reconstrucción del arquetipo \emph{X} --esto es, \emph{EM6} y \emph{PN1}--, a efectos prácticos de esta edición liminar se utilizó únicamente {\im} por ser el testimonio más representativo del subarquetipo, según lo estudiaron \textcite{PérezPriego2017} y \textcite{Beltrán1991,Beltrán2013}.}, la filiación de los testimonios se resumiría con el siguiente \emph{stemma} reducido a partir de lo anteriormente indicado\footnoteE{Para el \emph{stemma} completo \emph{cfr.} \textcite[103]{PérezPriego2017}.}:\par

	\begin{figure}[ht]
		\centering
		\begin{tikzpicture}[%
			level 1/.style={sibling distance=4cm},
			level 2/.style={sibling distance=2.5cm},
			level 3/.style={sibling distance=2.5cm},
			level 4/.style={sibling distance=2.5cm},
			edge from parent fork down,
			every node/.style={font=\normalsize}%
			]
			%
			\node {\emph{X}}
			child {node {α}
				child {node {LB3}}
				child {node {γ}
					child {node {δ}
						child {node {82*JM}}
						child [yshift=-25pt]{node {86*RL}}
					}
					child [yshift=-75pt] {node {ε}
						child {node {HH1}}
						child {node {MN19}}
					}
				}
			}
			child {node {β}
				child {node {83*IM}}
			};
		\end{tikzpicture}
	\end{figure}%

Se utilizó como texto base el testimonio {\eg} por portar este lecciones no contaminadas, conservar el castellano de la época, y por ser cercano a la redacción de las coplas.\par
%
Criterios de presentación gráfica\footnoteE{Las modificaciones no aplican al aparato de variantes, a excepción de \emph{nn} y las grafías dobles.}:
%
\begin{itemize}[label=--]%
\item \emph{b}--\emph{v} = se conserva la alternancia.
\item \emph{f} = se mantiene en vocablos como \comillas{fizo} [hizo] porque, en general, corresponden a la \emph{f} inicial latina y, por tanto anulan las sinalefas.
\item \emph{t} = se respeta la grafía en expresiones finalizadas en -\emph{at}, salvo cuando el testimonio exija una modificación por la rima.
\item \emph{x} = equivale a la fricativa posalveolar sorda, por tanto, se conserva.
\item \emph{ç} = se respetan en el texto las soluciones que ofrece el testimonio de base.
\item \emph{i}--\emph{y}--\emph{j} = \emph{i} se coloca para vocales; \emph{y} para la conjunción copulativa y para vocablos cuya grafía final es \emph{y} vocálico [rey]; \emph{j} solo para consonantes.
\item \emph{g}--\emph{j} = la alternancia se conserva.
\item \emph{v}--\emph{u} = se coloca \emph{v} cuando es una consonante y \emph{u} cuando es vocal.
\item \emph{rr}--\emph{ss}--\emph{ff} = se simplifican en \emph{r}, \emph{s} y \emph{f}.
\item \emph{nn} = se transcribe como \emph{ñ}.
\item \emph{th} = se conserva para indicar latinismos cultos en el texto.
\item \emph{⁊} = se interpreta por \emph{e}.%
\end{itemize}\par
%
Las contracciones se separan sin indicación alguna. Las grafías en cursiva corresponden, casi sin excepción, a siglas en {\eg}, que se desarrollaron según usos del copista; también cuando se adopten lecciones de {\hh} o de algún testimonio impreso.\par
%
El orden de las coplas que se sigue es el de {\eg} que, a su vez, coincide con el orden y conclusiones de \textcite{Senabre1983}, \textcite{Palumbo1983}, \textcite{Orduna1967}, \textcite{Beltrán1991,Beltrán2013}, \textcite{PérezPriego1990,PérezPriego2017} y \parencite{Foulché-Delbosc1902}, descartando el orden de {\jm}.\par
%
Constituyen al aparato crítico el orden y rúbricas en los testimonios, el aparato de variantes y las explicaciones adicionales. El tipo de aparato de variantes es negativo. Luego de indicar el pasaje al que corresponde la variante --si se coloca \comillas{\char"02016}, indicará que aún continúa en el mismo pasaje--, se colocará la lección adoptada seguida por el signo \comillas{]} que encierra la lección adoptada, y tras el signo, las lecciones descartadas seguidas por su(s) sigla(s)\footnoteE{Irán en cursiva las siglas y se separarán con comas cada una.}, separada cada unidad por un \comillas{;}. Se eludirán variantes gráficas y de espacio. Las explicaciones contendrán argumentos en torno a las lecciones adoptadas, problemas respecto al estado del testimonio de base y plausibles interpretaciones.\par
\relax

\section*{\centering \textsc{Bibliografía}}
\addcontentsline{toc}{section}{\textsc{Bibliografía}}
\nocite{*}
\printbibliography[heading=none]
\newpage

\section*{\centering \textsc{Texto y aparato críticos}}
\addcontentsline{toc}{section}{\textsc{Texto y aparato críticos}}
\newpage

\section*{\centering \textsc{Conspectvs siglorvm}}
\addcontentsline{toc}{section}{\textsc{Conspectvs siglorvm}}

\subsection*{\centering\normalsize \textsc{Sigla testimoniorvm}}
\addcontentsline{toc}{subsection}{\textsc{Sigla testimoniorvm}}
\vspace*{-2.5pt}
\begin{itemize}[label=,leftmargin=0.6cm]%
\item {\eg} = \emph{Cancionero de Egerton}, 1470?-1480?
\item {\jm} = \emph{Vita Christi por Fray Íñigo de Mendoza}, ca. 1480-1484.
\item {\rl} = \emph{Cancionero de Ramón de Llavia}, 1488-1490?
\item {\hh} = \emph{Cancionero de Oñate-Castañeda}, s. XV.
\item {\mn} = \emph{Cancionero de Pero Guillén de Segovia}, copia del s. XVIII de un MS. del s. XV.
\item {\im} = \emph{Cancionero de fray Íñigo de Mendoza}, 1483?%
\end{itemize}

\subsection*{\centering\normalsize \textsc{Sigla apparati critici}}
\addcontentsline{toc}{subsection}{\textsc{Sigla apparati critici}}
\vspace*{-2.5pt}
\begin{itemize}[label=,leftmargin=0.6cm]%
\item \emph{a. corr.} = ante correctionem.
\item \emph{gl.} = glossa.
\item \emph{corr.} = correxit.
\item \emph{eras.} = erasit.
\item \emph{err.} = erravit.
\item \emph{ras.} = rasura.
\item \emph{om.} = omisit.
\item \emph{p. corr.} = post correctionem.
\item \emph{pr.} = prius.
\item \emph{sec. m.} = secunda manus.
\item \emph{transp.} = transposuit.
\item \emph{δ} = {\jm} + {\rl}.
\item \emph{ε} = {\hh} + {\mn}.
\item \emph{γ} = \emph{δ} + \emph{ε}.%
\end{itemize} 
\newpage
%
\addcontentsline{toc}{section}{\emph{\textsc{Coplas}}}
\beginnumbering
\begin{center}
		[I]\footnoteA{[I] \textsc{Incipit} = Coplas que fiço don jorje manrriq\emph{ue} sobre la muerte del maestre de santiago don rrodrigo manrriq\emph{ue} su padre {\eg}; Dezir de don jorge manrriq\emph{ue} por la muerte de su padre {\jm}; De don jorge manrique por la muerte de su padre {\rl}; Coplas de do\emph{n} jorge ma\emph{n}riq\emph{ue} a la muerte del maestre don rrod\emph{rig}o ma\emph{n}riq\emph{ue} su padre {\hh}; Coplas que hizo do\emph{n} jorge manrriq\emph{ue} ala muerte del maestre de santiago do\emph{n} rodrigo manrique su padre {\im}.}
\end{center}
\pstart
\textsc{Recuerde}\footnoteE{\comillas{Recuerde} no tiene la acepción de \comillas{traer a la memoria}, sino la de \comillas{despertar} y, para ello, el \emph{Tesoro de la lengua castellana} de \textcite{Covarrubias1611} registra para el vocablo \emph{recordar} \comillas{despertar el que duerme, o boluer en acuerdo, del verbo recordor}. Posteriormente se registró el significado \comillas{Metaphoricamente vale despertar al que está dormido} (\emph{Autoridades}, 2ª acepción).} \textsc{el alma} dormida,\\
avive el seso y despierte\\
contenplando\\
cómo se pasa la vida,\\
cómo se viene la muerte\\
tan callando;\par
q\emph{uá}nd presto se va el plazer,\\
cómo después de acordado\\
da dolor,\\
cómo a nuestro paresçer\\
\edtext{q\emph{ua}lq\emph{uie}ra}{\Afootnote{qualqujer {\eg}}} tie\emph{n}po pasado \edtext{}{\Afootnote[nosep]{sapado \emph{err.} {\eg}}}\\
fue mejor.\par
\pend

\begin{center}
		[II]
\end{center}
\pstart
\edtext{Y pues}{\lemma{y pues}\Afootnote{Pues si \emph{δ}}} vemos lo presente\\
cómo en un pu\emph{n}to \edtext{se}{\Afootnote{\emph{om.} {\im}}} es ido\\
y acabado,\\
si judgamos sabiamente,\\
daremos lo no venido\\
por pasado.\par
No se engañe nadie, no,\\
pensando que ha de d\emph{e} durar\\ 
lo que espera,\\
más que duró lo q\emph{ue} vio,\\
pues que todo ha de pasar\\
por tal manera.\par
\pend

\begin{center}
		[III]\footnoteA{[III] conpara {\eg}.}
\end{center}
\pstart
N\emph{uest}ras vidas son los ríos\\
que van a dar en la mar\\
que es el morir;\\
allí van los señoríos\\
derechos a se acabar\\
y consumir.\par
Allí los ríos cabdales,\\
allí los otros, medianos,\\
y más chicos,\\
\edtext{y llegados}{\Afootnote{alleguados {\jm}; allegados {\im}}}\footnoteE{\textcite{PérezPriego2017} sugiere que la lección \emph{alleguados}, probablemente a través de la contaminación de {\jm} al subarquetipo \emph{β}, habría generado la lección \emph{allegados}.}, son iguales,\\
los que bive\emph{n} por sus manos\\
y los ricos.\par
\pend

\begin{center}
	[IV]\footnoteA{[IV] ynuoca {\eg}; inuocaçion {\jm}; Inuocacion {\rl}.}
\end{center}
\pstart
Dexo las invocaçiones\\
de los famosos poetas\\
y oradores;\\
no \edtext{curo}{\Afootnote{cureys {\eg}}} de sus \edtext{ficçiones}{\Afootnote{afficciones {\jm}}},\\
\edtext{que traen}{\Afootnote{porque traen {\hh}}} yervas secretas\\
sus sabores.\par
Aquel solo me encomie\emph{n}do,\\
aq\emph{ue}l solo invoco yo \edtext{}{\Afootnote[nosep]{de verdat \emph{a. corr.} {\eg}}}\\
de verdad\footnoteE{En {\eg} se redactó \comillas{[de ver] dat} luego de \comillas{ynuoco yo}. Quizá el redactor inicial --o un secundario [\emph{cfr. gl.} XL]-- tachó el fragmento por la errónea ubicación y corrigió \comillas{de verdad} en un pequeño espacio entre los vv. 44 y 46. Puesto que se modificó \emph{verdat} por \emph{verdad} en la corrección, para la fijación se reemplazó \emph{t} en \emph{deydat} por \emph{d} para conservar la rima.} \edtext{}{\Afootnote[nosep]{\emph{p. corr.} {\eg}}},\\
que en este mu\emph{n}do \edtext{bivie\emph{n}do}{\lemma{biuie\emph{n}do}\Afootnote{vinjendo {\hh}}},\\
el mu\emph{n}do no conosçió\\
su deidad.\par
\pend

\begin{center}
	[V]\footnoteA{[V] ap[lica] y [co]\emph{n}para \emph{ras.} {\eg}.}
	\end{center}
\pstart
Este mu\emph{n}do es el camino\\
para el otro, que es morada\\
sin pesar,\\
mas cunple tener bue\emph{n} tino\\
para andar \edtext{esta jornada}{\Afootnote{este camino {\jm}}}\\
sin errar.\par
Partimos q\emph{ua}ndo nasçemos,\\
andamos \edtext{q\emph{ua}ndo}{\Afootnote{mientra \emph{δ}; qua\emph{n}to {\hh}}} bivimos\\
y \edtext{llegamos}{\Afootnote{allegamos {\im}}}\\
al tienpo que feneçemos;\\
así que, q\emph{ua}ndo morimos,\\
descansamos.\par
\pend

\begin{center}
	[VI]\footnoteA{[VI] p\emph{r}osigue {\eg}.}
\end{center}
\pstart
\edtext{Este}{\lemma{este}\Afootnote{y este {\jm}}} mu\emph{n}do bueno fue\\
si bie\emph{n} \edtext{usásemos}{\lemma{vsasemos}\Afootnote{vsamos {\hh}}} de él\\
como devemos,\\
porque segu\emph{n}d n\emph{uest}ra fe,\\
es para ganar aquel\\
que atendemos;\par
\edtext{y}{\Afootnote{\emph{om.} {\rl}}} aun aq\emph{ue}l fijo de Dios,\\
para sobirnos al çielo\\
desçendió\\
a nasçer \edtext{acá}{\lemma{aca}\Afootnote{aqua {\jm}}} entre nos\\
\emph{e} bevir en este suelo\\
do murió.\par
\pend

\begin{center}
	[VII]\footnoteA{[VII] XXV en {\jm}; XIII en {\rl}.}
\end{center}
\pstart
Si fuese en n\emph{uest}ro poder\\
\edtext{tornar}{\Afootnote{hazer {\rl}}} la cara fermosa\\
corporal\\
como podemos fazer\\
al \edtext{ánima}{\lemma{anima}\Afootnote{alma {\rl}, {\hh}}} \edtext{glorïosa}{\lemma{gloriosa}\Afootnote{tan gloriosa {\rl}, {\hh}}}\\
angelical,\par
¡qué dilige\emph{n}çia ta\emph{n} biva\\
\edtext{toviéramos}{\lemma{touieramos}\Afootnote{terniamos {\jm}}} toda ora\\
y tan presta\\
en conponer la cativa,\\
dexándonos la señora\\
desconpuesta!\par
\pend
\vfill
\relax
\newpage

\begin{center}
	[VIII]\footnoteA{[VIII] VII en \emph{86*JM}.}
\end{center}
\pstart
Ved de q\emph{uá}nd poco balor\\
son las cosas tras q\emph{ue} andamos\\
y corremos\\
que, en este mu\emph{n}do traidor,\\
\edtext{aun}{\lemma{avn}\Afootnote{cavn {\hh}}} p\emph{ri}m\emph{er}o que muramos\\
las perdemos:\par
de ellas desfaze la hedad,\\
de ellas, casos desastrados\\
q\emph{ue} \edtext{acaesçen}{\Afootnote{contecen {\im}}},\\
\edtext{y de ellas, por calidad}{\lemma{y dellas por calidad}\Afootnote{dellas por su calidad \emph{δ}, {\hh}, {\im}}},\\
en los más altos estados\\
desfallesçen.\par
\pend

\begin{center}
	[IX]\footnoteA{[IX] VIII en {\jm}, {\rl}.}
\end{center}
\pstart
\edtext{Dezidme}{\lemma{dezidme}\Afootnote{Dezimos {\im}}}, la fermosura,\\
la gentil frescura y tez\\
de la cara,\\
la color y la blancura,\\
q\emph{ua}ndo viene la vejez,\\
¿quál se para?\par
Las \edtext{mañas}{\Afootnote{manos {\jm}; maneras {\rl}}} y ligereza\\
y la fuerça corporal\\
de joventud,\\
todo se torna graveza\\
q\emph{ua}ndo llega al arraval \edtext{}{\Afootnote[nosep]{a la hedat \emph{err.} {\eg}}}\\
de senetud.\par
\pend

\begin{center}
	[X]\footnoteA{[X] IX en {\jm}, {\rl}.}
\end{center}
\pstart
Pues la sangre d\emph{e} los godos,\\
y el linaje y la nobleza\\
tan creçida,\\
¡por q\emph{uá}ntas vías y modos\\
se sume su grande alteza\\
en esta vida!\par
Unos, por poco \edtext{valer}{\Afootnote{valor {\hh}}},\\
¡por q\emph{uá}nd \edtext{baxos}{\Afootnote{varios {\jm}; baros {\rl}}} y abatidos\\
que los tiene\emph{n}!\\
\edtext{Y otros, por poco tener}{\lemma{y otros por poco}\Afootnote{yaotros por no {\jm}; otros por no {\rl}; y otros por no {\hh}; otros que por no {\mn}, {\im}}},\\
con oficios no devidos\\
se \edtext{ma\emph{n}tiene\emph{n}}{\Afootnote{sostienen {\im}}}.\par
\pend

\begin{center}
	[XI]\footnoteA{[XI] X en {\jm}, {\rl}.}
\end{center}
\pstart
Los estados y \edtext{riqueza}{\Afootnote{riquezas \emph{δ}, {\hh}}}\\
que nos \edtext{dexen}{\Afootnote{dexan {\jm}}} a deshora,\\
¿quié\emph{n} \edtext{lo}{\Afootnote{\emph{om.} {\eg}}} dubda?\\
\edtext{No}{\lemma{no}\Afootnote{nos {\rl}}} les pidamos \edtext{firmeza}{\Afootnote{firmezas {\jm}, {\hh}}},\\
\edtext{pues que}{\Afootnote{por que {\jm}}} son de una señora\\
que se muda:\par
que bienes son de Fortuna\\
que \edtext{rebuelve}{\lemma{rebuelue}\Afootnote{rebueluen \emph{δ}}} co\emph{n} su rueda\\
presurosa,\\
la qual no puede ser una\\
ni \edtext{estar}{\Afootnote{ser {\im}}} estable ni q\emph{ue}da\\
en una cosa.\par
\pend

\begin{center}
	[XII]\footnoteA{[XII] XI en {\jm}, {\rl}.}
\end{center}
\pstart
\edtext{Y pues}{\lemma{y pues}\Afootnote{pero \emph{γ}, {\im}}} digo que \edtext{acompañen}{\Afootnote{aco\emph{n}pañe {\eg}}}\\
y \edtext{lleguen}{\Afootnote{llegue {\eg}}} fasta la huesa\\
co\emph{n} su dueño;\\
\edtext{por eso}{\Afootnote{por esto {\jm}; mas por eso {\hh}}} \edtext{no nos engañen}{\Afootnote{no nos engañe {\eg}; no sengañen {\hh}}}\footnoteE{Se adoptaron las lecciones plurales del texto mayoritario frente a las de número singular --esto es, \comillas{aco\emph{n}pañe}, \comillas{llegue} y \comillas{engañe}-- que porta {\eg}.},\\
pues se va la vida \edtext{apriesa}{\Afootnote{espesa {\mn}}}\\
como sueño.\par
\edtext{Y}{\lemma{y}\Afootnote{q\emph{ue} {\hh}}} los \edtext{plazeres}{\Afootnote{deleytes \emph{γ}, {\im}}} de acá\\
son, en q\emph{ue} nos deleitamos,\\
tenporales,\\
y los torme\emph{n}tos de allá,\\
que por ellos esperamos,\\
eternales.\par
\pend

\begin{center}
	[XIII]\footnoteA{[XIII] XII en {\jm}, {\rl}.}
\end{center}
\pstart
Los plazeres \emph{e} dulçores\\
de esta vida \edtext{trabajada}{\Afootnote{trabajosa {\eg}}}\\
que tenemos\\
\edtext{que son}{\Afootnote{no son {\rl}, {\im}}} sino corredores,\\
y la muerte, la çelada\\
en que caemos.\par
No mirando \edtext{n\emph{uest}ro daño}{\Afootnote{nuestros daños {\mn}}},\\
corremos a rie\emph{n}da suelta\\
sin parar;\\
\edtext{desque}{\Afootnote{cuando {\hh}, {\im}}} vemos el engaño,\\
\edtext{si}{\Afootnote{y \emph{γ}, {\im}}} queremos dar la buelta\\
no ay lugar.\par
\pend

\begin{center}
	[XIV]\footnoteA{[XIV] XXVI en {\jm}.}
\end{center}
\pstart
Estos reyes poderosos\\
que \edtext{vemos}{\lemma{veemos}\Afootnote{leemos {\jm}; vedes \emph{ε}}} por escrituras\\
ya pasadas,\\
co\emph{n} \edtext{casos tristes}{\Afootnote{talos frjos {\rl}}}, llorosos,\\
fuero\emph{n} sus buenas ve\emph{n}turas\\
\edtext{trastornadas}{\Afootnote{acabadas {\hh}}};\par
así que no ay cosa fuerte,\\
\edtext{que papas y enperadores}{\Afootnote{reyes papas emperadores {\rl}; a papas ni emperadores {\hh}; que a papas y emperadores {\im}}}\\
y \edtext{perlados}{\Afootnote{prelados {\jm}}},\\
así los trata la muerte\\
como a los pobres pastores\\
de ganados.\par
\pend

\begin{center}
	[XV]\footnoteA{[XV] XXVII en {\jm}.}
\end{center}
\pstart
Dexemos a los troyanos,\\
que sus males no los vimos\\
ni sus glorias;\\
dexemos a los romanos,\\
\edtext{aunque leemos y oímos}{\lemma{avn que leemos y oymos}\Afootnote{avn q\emph{ue} oymos y leemos \emph{δ}; aunque oymos y leimos \emph{ε}, {\im}}}\\
\edtext{sus estorias}{\Afootnote{sus victorias {\jm}, {\hh}, {\im}}}.\par
No curemos de saber\\
lo de aquel \edtext{siglo}{\Afootnote{tie\emph{m}po {\hh}}} pasado\\
qué fue de ello;\\
vengamos a lo de ayer\\
que tan bie\emph{n} es olvidado\\
como aquello.\par
\pend

\begin{center}
	[XVI]\footnoteA{[XVI] XXVIII en {\jm}.}
\end{center}
\pstart
¿Qué se fizo el rey don Jua\emph{n}?\\
Los infantes de Aragó\emph{n},\\
¿qué se fiziero\emph{n}?\\
¿Qué fue de ta\emph{n}to galán?\\
¿Qué fue de tanta inve\emph{n}çión\\
como \edtext{traxiero\emph{n}}{\Afootnote{truxieron \emph{δ}, {\mn}}}?\par
Las justas y los torneos,\edtext{}{\linenum{|||||189}\Afootnote[nosep]{\hspace*{-0.4em}/190–192 \space\emph{transp.} {\rl}}}\\
parame\emph{n}tos, bordaduras\\
y çimeras\\
¿fuero\emph{n} sino devaneos?\\
¿Qué fuero\emph{n} sino verduras\\
de las eras?\par
\pend

\begin{center}
	[XVII]\footnoteA{[XVII] XXIX en {\jm}.}
\end{center}
\pstart
¿Qué se fiziero\emph{n} las damas,\\
sus tocados, sus vestidos,\\
sus olores?\\
¿Qué se fiziero\emph{n} las llamas\\
de los fuegos ençe\emph{n}didos\\
\edtext{de amadores}{\Afootnote{de amores {\eg}, {\jm}}}?\par
¿Qué se fizo aquel trobar,\\
las músicas acordadas\\
que tañían?\\
¿Qué se fizo aquel dançar,\\
aquellas ropas \edtext{trepadas}{\Afootnote{japadas {\jm}; chapadas {\rl}, \emph{ε}, {\im}}}\\
que \edtext{traían}{\lemma{trayan}\Afootnote{vestian \emph{ε}}}?\par
\pend

\begin{center}
	[XVIII]\footnoteA{[XVIII] XXX en {\jm}.}
\end{center}
\pstart
Pues el otro, su heredero,\\
don Enriq\emph{ue}, ¡q\emph{ué} \edtext{poderes}{\Afootnote{poderoso {\jm}}}\\
alcançava!\\
¡Quán blando y q\emph{uá}nd falaguero,\\
el mu\emph{n}do co\emph{n} sus plazeres\\
se le dava!\par
Mas \edtext{verás}{\lemma{veras}\Afootnote{vereys {\jm}}} q\emph{uá}nd enemigo,\\
q\emph{uá}nd contrario, q\emph{uá}nd cruel\\
se le mostró;\\
aviéndole seído\footnoteE{\comillas{seydo} proviene del vocablo aragonés \comillas{seyer} (\emph{DICCA-XV}). En cuanto al conteo silábico, probablemente contiene una sinéresis, porque en su acentuación regular se cuenta \comillas{se-ý-do}, mas, por ser \emph{sido} una trivialización, según porta la lección {\jm}, \emph{seydo}, por cuestiones métricas, para no causar un eneasílabo, debe necesariamente contener una sinéresis en \comillas{seý}, calculando así el octosílabo \comillas{a-vién-do-le-seý-do a-mi-go}.} amigo,\\
¡q\emph{uá}nd poco duró con él\\
lo que le dio!\par
\pend

\begin{center}
	[XIX]\footnoteA{[XIX] XXXI en {\jm}.}
\end{center}
\pstart
Las dádivas \edtext{desmedidas}{\Afootnote{demasiadas {\im}}},\\
los hedefiçios reales\\
llenos de oro,\\
las baxillas ta\emph{n} febridas,\\
los enriques y reales\\
del thesoro,\par
los jaezes, los cavallos\\
\edtext{de su gente}{\Afootnote{desta gente {\jm}; de sus gentes {\rl}; y sus gentes {\hh}}}\edtext{}{\Afootnote[nosep]{\emph{om.} {\mn}}} y atavíos\\
ta\emph{n} sobrados\\
¿\edtext{dónde}{\lemma{donde}\Afootnote{ado {\hh}}} iremos a buscallos?\\
¿Qué fuero\emph{n} sino roçíos\\
de los prados?\par
\pend

\begin{center}
	[XX]\footnoteA{[XX] XXXII en {\jm}.}
\end{center}
\pstart
Pues su herm\emph{an}o el inoçe\emph{n}te,\\
que en su vida suçesor\\
\edtext{se llamó}{\lemma{se llamo}\Afootnote{le firieron {\rl}}},\\
qué corte ta\emph{n} exçelente\\
tovo, y q\emph{uá}nto gran señor\\
\edtext{le}{\Afootnote{se {\jm}}} \edtext{siguió}{\lemma{siguio}\Afootnote{siguieron {\rl}}};\par
mas como fuese mortal,\\
\edtext{metiólo}{\lemma{metiolo}\Afootnote{leuo le {\jm}; echole {\hh}}} la muerte luego\\
\edtext{en su}{\Afootnote{en la su {\jm}}} fragua.\\
¡Oh, juïzio divinal,\\
q\emph{ua}ndo más ardía el fuego,\\
\edtext{echaste}{\Afootnote{cebaste {\mn}}} agua!\par
\pend

\begin{center}
	[XXI]\footnoteA{[XXI] XXXIII en {\jm}.}
\end{center}
\pstart
Pues aquel gran Co\emph{n}destable,\\
\edtext{maestre}{\Afootnote{maestro {\jm}}} que conosçimos\\
ta\emph{n} privado,\\
\edtext{no cunple}{\Afootnote{que cu\emph{m}ple {\hh}}} \edtext{q\emph{ue} de él se fable}{\lemma{q\emph{ue} del se fable}\Afootnote{que mas hable {\rl}}},\\
\edtext{sino sólo que lo}{\lemma{sino solo que lo}\Afootnote{sino solo lo que {\jm}; mas solo como lo {\rl}; saluo solo que lo {\hh}}} vimos\\
degollado;\par
sus infinitos tesoros,\\
sus villas \edtext{y}{\Afootnote{\emph{om.} {\rl}}} sus lugares,\\
su ma\emph{n}dar,\\
¿qué le fuero\emph{n} sino lloros?\\
¿\edtext{Qué fuero\emph{n} sino pesares}{\lemma{que fuero\emph{n} sino pesares}\Afootnote{dolores, ansias pesares {\mn}; fueron le sy no {\im}}}\\
al dexar?\par
\pend

\begin{center}
	[XXII]\footnoteA{[XXII] XXXIV en {\jm}.}
\end{center}
\pstart
\edtext{Pues}{\lemma{pues}\Afootnote{E {\rl}}} los otros dos herm\emph{an}os,\\
maestres ta\emph{n} prosperados\\
como reyes\\
que a los \edtext{gra\emph{n}des}{\Afootnote{baxos {\hh}}} y medianos\\
troxiero\emph{n} tan sojudgados\\
a sus leyes;\par
\edtext{aquella}{\Afootnote{y su grand {\mn}}} prosperidat\\
que e\emph{n} ta\emph{n} alto fue sobida\\
y ensalçada;\\
¿qué fue sino claridat,\\
\edtext{que}{\Afootnote{quando {\rl}}} estando más ençe\emph{n}dida\\
fue \edtext{amatada}{\Afootnote{matada {\jm}}}?\par
\pend
\relax
\vfill
\newpage

\begin{center}
	[XXIII]\footnoteA{[XXIII] XXXV en {\jm}.}
\end{center}
\pstart
Tantos duq\emph{ue}s exçele\emph{n}tes,\\
tantos marq\emph{ue}ses y co\emph{n}des\\
y varones\\
como vimos ta\emph{n} pote\emph{n}tes,\\
di, Muerte, ¿do los asco\emph{n}des\\
\edtext{y traspones}{\Afootnote{los pones {\jm}}}?\par
\edtext{Y las sus}{\lemma{y las sus}\Afootnote{y sus muy {\im}}} claras hazañas\\
que fiziero\emph{n} en las guerras\\
y en las pazes,\\
q\emph{ua}ndo tú, cruda\footnoteE{\comillas{cruda} = \comillas{Se toma tambien por cruel, áspero, sangriento y desapiadado.} (\emph{Autoridades}, 3ª acepción).}, te ensañas,\\
co\emph{n} tu fuerça las atierras\\
y desfazes\footnoteE{\comillas{guerras/pazes}, \comillas{atierras/desfazes} = es probable que {\jm} haya leído erróneamente, invirtiendo el orden entre los vv. 272-273 y 275-276, algo que puede conjeturarse por la lección \emph{passos}, quizá una lectura errónea de \emph{pazes} y \emph{guerras}, que aparece en 273, cuando en el resto de testimonios aparece en el 272, al igual que en lugar de \emph{atierras} lee \emph{desfazes} y al revés.}.\par
\pend

\begin{center}
	[XXIV]\footnoteA{[XXIV] XXXVI en {\jm}.}
\end{center}
\pstart
Las \edtext{huestes}{\Afootnote{\emph{om.} {\mn}}} innumerables,\\
los pendones y esta\emph{n}dartes\\
y vanderas,\\
los castillos inpunables,\\
los muros y baluartes\\
y barreras,\par
la cava ho\emph{n}da, \edtext{chapada}{\Afootnote{japada {\jm}}},\\
\edtext{y}{\Afootnote{o \emph{γ}, {\im}}} qualquier otro reparo\\
¿qué aprovecha?\\
\edtext{Q\emph{ua}ndo}{\lemma{q\emph{ua}ndo}\Afootnote{que si \emph{ε}, {\im}}} tú vienes irada,\\
todo lo \edtext{pasas}{\Afootnote{lleuas {\hh}}} \edtext{de claro}{\Afootnote{de claro en claro {\jm}}}\\
co\emph{n} tu flecha.\par
\pend

\begin{center}
	[XXV]\footnoteA{[XXV] XIII en {\jm}. || dirige la fabla al maestre do\emph{n} rodrigo su padre {\eg}; ffabla del maestre do\emph{n} rr\emph{odri}go ma\emph{n}riq\emph{ue} {\hh}.}
\end{center}
\pstart
Aquel de \edtext{buenos}{\Afootnote{bienes {\mn}}} abrigo,\\
amado por virtuoso\\
de la gente,\\
el maestre don Rodrigo\\
Ma\emph{n}riq\emph{ue}, \edtext{tanto}{\Afootnote{tan {\hh}, {\im}}} \edtext{famoso}{\Afootnote{fermoso {\jm}}},\\
\edtext{tan}{\Afootnote{y tan {\eg}, {\im}}} valiente,\par
sus grandes fechos y \edtext{claros}{\Afootnote{actos {\jm}}}\\
no cu\emph{n}ple que los \edtext{alabe}{\Afootnote{acaben {\mn}}},\\
pues los viero\emph{n},\\
ni los quiero fazer caros,\\
\edtext{pues que todo el mu\emph{n}do sabe}{\Afootnote{pues que todo el mu\emph{n}do lo sabe {\eg}; pues quel mundo todo sabe {\rl}; pues el mundo todo sabe {\im}}}\\
quales fuero\emph{n}.\par
\pend

\begin{center}
	[XXVI]\footnoteA{[XXVI] XIV en {\jm}.}
\end{center}
\pstart
¡\edtext{Qué amigo de sus amigos}{\lemma{Que amigo desus amigos}\Afootnote{que amigo de amigos {\eg}, {\hh}; amigo de sus amigos {\mn}, {\im}}}!,\\
¡qué señor pa\emph{ra} criados\\
y parientes!\\
Qué enemigo de enemigos!\\
¡Q\emph{ué} \edtext{maestro}{\Afootnote{maestre {\eg}}} de esforçados\\
y valientes!\par
Q\emph{ué} seso para discretos!\\
Qué gr\emph{aci}a para donosos!\\
¡Qué razón!\\
¡Qué benigno a los sugebtos!\\
Y a los bravos \edtext{y dañosos}{\Afootnote{y furiosos {\jm}; y soberuios {\hh}}},\\
¡un leó\emph{n}!\par
\pend

\begin{center}
	[XXVII]\footnoteA{[XXVII] XV en {\jm}. || conpara {\eg}.}
\end{center}
\pstart
En \edtext{}{\Afootnote[nosep]{en [la] \emph{eras.} {\eg}}} ventura, \edtext{Otavïano}{\lemma{otauiano}\Afootnote{octoniano {\jm}}},\\
Jullo Çésar en vençer\\
y batallar;\\
en la virtud, Africano,\\
Aníbal en el saber\\
y trabajar.\par
En la bondad, un \edtext{Trajano}{\lemma{trajano}\Afootnote{hatrajano {\rl}}},\\
Tito en liberalidad\\
con alegría;\\
\edtext{en su braço}{\Afootnote{enla nobleza {\hh}}}, \edtext{Aurelïano}{\lemma{aureliano}\Afootnote{hun archiano {\jm}, {\im}; abreliano {\rl}; el troyano {\hh}}},\\
\edtext{Marco Atilio}{\lemma{marco atilio}\Afootnote{marco tulio {\eg}, {\im}; mateo tulio {\jm}}} en la verdad\\
que prometía.\par
\pend

\begin{center}
	[XXVIII]\footnoteA{[XXVIII] XXVI en {\jm}. || prosigue {\eg}.}
\end{center}
\pstart
Antonio \edtext{Pío}{\lemma{pio}\Afootnote{pia {\jm}}} e\emph{n} cleme\emph{n}çia,\\
\edtext{Marco Aurelio}{\lemma{marco aurelio}\Afootnote{arebo anthonio {\jm}}} e\emph{n} ig\emph{ua}ldat\\
\edtext{y}{\Afootnote{del \emph{δ}, {\mn}, {\im}; y bue\emph{n} {\hh}}} senblante;\\
\edtext{Adrïano}{\lemma{adriano}\Afootnote{adriapo \emph{86*JM}}} en eloq\emph{ue}nçia,\\
\edtext{Theodosio}{\lemma{theodosio}\Afootnote{theosio {\eg}; verdosio {\rl}}} en umanidat\footnoteE{Cuenta con 9 sílabas por las sílabas gramaticales más la aguda \emph{–dad}. Quilis (2009) sostiene que aquello corresponde a una sinéresis en «Theo», leyéndose así «Theo-do-sioen hu-ma-ni-dad».}\\                                                            
y buen talante.\par
\edtext{Aurelio Alexandre}{\lemma{aurelio alexandre}\Afootnote{marco alixandre {\eg}; aurelio y alexandre {\jm}; aurelto alexandre {\rl}; aurelio liandre {\hh}}} fue\\
en diçiplina y rigor\\
de la guerra;\\
\edtext{un}{\Afootnote{\emph{om.} {\eg}}} Constantino en la fe,\\
\edtext{Camillo}{\lemma{camjllo}\Afootnote{canjno {\eg}; camelio {\im}}} en el \edtext{gran}{\Afootnote{gra\emph{n}de {\hh}}} amor\\
de su tierra.\par
\pend

\begin{center}
	[XXIX]\footnoteA{[XXIX] XVII en {\jm}; XXX en {\hh}.}
\end{center}
\pstart
No dexó grandes thesoros\\
ni \edtext{alcançó}{\lemma{alcanço}\Afootnote{llego {\hh}}} \edtext{gra\emph{n}des}{\Afootnote{muchas {\rl}}} riq\emph{ue}zas\\ 
ni baxillas,\\
mas fizo \edtext{guerras a moros}{\Afootnote{guerra a los moros \emph{δ}, {\mn}, {\im}; gran guerra a moros {\hh}}}\footnoteE{Se siguen las lecciones plurales de {\eg}, porque don Rodrigo Manrique probablemente no participó en una guerra, sino en otras más. Es necesario atender a los datos proporcionados por Fernando del Pulgar en su \emph{Claros varones de Castilla} [ca. 1480-1486] que, en un capítulo dedicado al maestre, el XIII, titulado «El Maestre Don Rodrigo Manrrique, Conde de Paredes» --que ya citó \textcite[cap. IX]{MenendezYPelayo1944}--, los recoge con gran detalle. A continuación, el pasaje según la edición crítica de Tate:\par
«\emph{En las batallas ⁊ muchos recuentros que ovo con moros ⁊ con christianos este cavallero fue el que, mostrando grand esfuerço a los suyo, fería primero en los contrarios. E las gentes de su conpañía, visto el efuerço de su capitán, todos le seguían ⁊ cobravan osadía de pelear. Tenía tan grand conoscimiento de las cosas del canpo, ⁊ proveí[a]las en tal manera que donde él fue principal capitán nunca puso su gente en logar do se oviese de retraer, porque bolver las espaldas al enemigo era tan ageno que su ánimo que elegía antes de recebir la muerte peleando que salvar la vida huyendo.}\par 
»\emph{Este cavallero osó acometer grandes fazañas. Especialmente escaló una noche la cibdad de Huesca, que es del reino de Granada. E como quier que subiendo el escala los suyos fueron sentidos de los moros, ⁊ fueron algunos derribados del adarve ⁊ feridos en la subida} [...] \emph{subieron el muro peleando ⁊ no fallescieron de sus fuerças defendiéndolo, aunque veían los unos derramar su sangre, los otros caer de la cerca. E de esta manera, matando de los moros ⁊ muriendo de los suyos, este capitán, ferido en el braço de una saeta, peleando entró la cibdad e retroxo los moros fasta que los cercó en la fortaleza.} [...] \emph{Este cavallero duró ⁊ fizo durar a los suyos conbatiendo los moros que tenía cercados, e resistiendo a los moros que le tenía cercado por espacio de dos días, fasta que vino el socorro }[...] \emph{Ganada aquella cibdad, dexado en ella por capitán a un su hermano llamado Gómes Manrrique, ganó otras fortalezas en la comarca. Socorrió muchas vezes algunas cibdades ⁊ villa ⁊ capitanes christianos en tienpo de estrema necesidad. ⁊ fizo tanta guerra en aquellas tierras que en el reino de Granada el nonbre de Rodrigo Ma[n]rrique fue mucho tienpo a los moros grand terror.}»}\\
gana\emph{n}do sus fortalezas \edtext{}{\Afootnote[nosep]{fortalezas [y villas] \emph{eras.} {\eg}}}\\
y sus villas;\par
y en las lides que vençió,\\
\edtext{muchos moros y cavallos}{\Afootnote{quantos moros ⁊ cauallos {\rl}; muchos caualleros y cauallos {\im}}}\\                                      
se perdiero\emph{n},\\
\edtext{y}{\Afootnote{\emph{om.} {\hh}}} en este ofiçio ganó\\
las \edtext{rentas}{\Afootnote{villas {\eg}; rendas {\rl}}} y los basallos\\
que le diero\emph{n}.\par
\pend 

\begin{center}
	[XXX]\footnoteA{[XXX] XVIII en {\jm}; XXIX en {\hh}.}
\end{center}
\pstart
Pues \edtext{por su}{\Afootnote{ensu {\jm}}} ho\emph{n}rra y estado,\\
\edtext{en otros}{\Afootnote{en estos {\eg}}} tie\emph{n}pos pasados,\\
¿cómo se \edtext{uvo}{\lemma{vuo}\Afootnote{vio {\mn}}}?\\ Quedando desanparado,\\
\edtext{con hermanos}{\Afootnote{con sus fijos {\eg}}} y criados\\
se sostuvo.\par
Después q\emph{ue} fechos famosos\\
fizo \edtext{en estas dichas guerras}{\Afootnote{en la dicha guerra {\jm}; en esta misma guerra {\rl}; en esta dicha guerra \emph{ε}, {\im}}}\\
que fazía,\\
fizo tratos ta\emph{n} ho\emph{n}rrosos\\
que le diero\emph{n} \edtext{aún}{\lemma{avn}\Afootnote{muy {\im}}} más \edtext{t\emph{ie}rras}{\Afootnote{tierra \emph{γ}, {\im}}}\\
\edtext{que}{\Afootnote{quel {\jm}}} tenía.\par
\pend

\begin{center}
	[XXXI]\footnoteA{[XXXI] XIX en {\jm}.}
\end{center}
\pstart
\edtext{Y}{\lemma{y}\Afootnote{\emph{om. γ}, {\im}}} estas sus viejas istorias\\
que co\emph{n} su braço \edtext{pintó}{\lemma{pinto}\Afootnote{gano {\hh}}}\\
en \edtext{la}{\Afootnote{\emph{om.} {\rl}, {\hh}}} joventud,\\
co\emph{n} otras nuevas bitorias,\\
agora las renovó\\
en \edtext{la}{\Afootnote{\emph{om.} {\rl}, {\hh}}} senetud\footnoteE{363, \comillas{en la joventud}; 366, \comillas{en la senetud} = ambos pasajes, siguiendo a {\eg} y {\jm}, son un pentasílabo agudo, esto es, un hexasílabo, incurriendo, evidentemente, en una hipermetría. En este caso, {\rl} y {\hh} omiten \emph{la}, pero aun así incurren en una hipermetría por ser tetrasílabos agudos, por tanto, pentasílabos. \textcite{Beltrán1991} selecciona \emph{en la juventud}, lección de su testimonio de base, pero en \textcite{Beltrán2013} selecciona \emph{en juventud}, y no ofrece argumentos de su cambio, sino que lo descarta en el aparato crítico. Según Navarro, en los pasajes «irregulares» Manrique aplicó el pentasílabo quebrado «en condiciones de compensación o sinalefa respecto al octosílabo precedente», que fue «meramente potestativa» y que «no la aplicó de manera sistemática» \parencite[175]{Navarro1961}. Aun así, estos dos pasajes parecen ser irresolubles, incluso aplicando esta regla, que solo funciona si se seleccionan las variantes de {\rl} y {\hh}, esto es, omitir \emph{la}. Resulta complejo cuando otros testimonios no corrigen la hipermetría con sus lecciones. Los lectores podrán elegir cuál variante consideran adecuada, en tanto no sea la de {\mn}, que ya «excede el exceso» con su hexasílabo agudo.};\par
\edtext{por}{\Afootnote{que por {\eg}}} su grand abilidat,\\
\edtext{por}{\Afootnote{y por {\eg}}} méritos y ançianía\\
bie\emph{n} gastada,\\
\edtext{alcançó}{\lemma{alcanço}\Afootnote{alçando {\hh}}} la \edtext{dignidat}{\Afootnote{diuinidad {\jm}}}\\
la grand cavall\emph{er}ía\\
del espada.\par
\pend

\begin{center}
	[XXXII]\footnoteA{[XXXII] XX en {\jm}.}
\end{center}
\pstart
\edtext{Y}{\lemma{y}\Afootnote{\emph{om.} {\jm}}} sus villas \edtext{y sus t\emph{ie}rras}{\Afootnote{y lugares {\mn}}}\\
ocupadas de tiranos\\
las falló,\\
mas por çercos y por guerras\\
y por fuerça de sus manos\\ 
las \edtext{cobró}{\lemma{cobro}\Afootnote{gano {\hh}}}.\par
Pues nuestro rey natural,\edtext{}{\linenum{|||||380}\Afootnote[nosep]{\emph{transp.} {\eg} = \comillas{si de las obras que obro / el nuestro Rey natural}}}\\
si de las obras que obró\\
fue servido,\\
dígalo el de Portugal,\\
\edtext{y en}{\Afootnote{en {\eg}; si en {\rl}; quien en {\hh}}} Castilla, quien siguió\\
su partido.\par                                                                          
\pend

\begin{center}
	[XXXIII]\footnoteA{[XXXIII] XXI en {\jm}.}
\end{center}
\pstart
Después \edtext{que puso}{\Afootnote{de puesta {\eg}, \emph{δ}}} la vida\\
tantas vezes por su ley\\
al tablero,\\
después de ta\emph{n} bie\emph{n} servida\\
la corona de su rey\\                                                                         
verdadero,\par
despu\emph{é}s de ta\emph{n}ta hazaña\\
a que no puede bastar\\ 
cuenta çierta,\\
en la su villa de Ocaña,\\
vino la Muerte \edtext{a}{\Afootnote{\emph{om.} {\im}}} llamar\edtext{}{\Afootnote[nosep]{lamar \emph{err.} {\im}}}\\
a su puerta,\par
\pend

\begin{center}
	[XXXIV]\footnoteA{[XXXIV] XXII en {\jm}. || ffabla la muerte con el maestre {\eg}.}
\end{center}
\pstart
diziendo: «bue\emph{n} cavallero,\\
dexad el mu\emph{n}do engañoso\\                                                               
\edtext{sin falago}{\Afootnote{y su [\emph{f}-\emph{h}]alago \emph{γ}; con halago {\im}}}\footnoteE{Los demás testimonios transmiten «y su [\emph{f}-\emph{h}]alago», que parece tener más sentido, pero que incurre en una hipermetría, porque \emph{h} corresponde a una consonante aspirada o una \emph{f} inicial latina, ya que entre los testimonios alternan \emph{h} [{\rl}] y \emph{f} [{\hh}, {\mn}], aunque alguno la omite [{\jm}]; de ser así esto, algo apreciable por la inestabilidad gráfica, la sinalefa no sería posible y, por tanto, sería conveniente adoptar la lección de {\eg}, que es un tetrasílabo. La lección \emph{sin falago} apuntaría --probablemente-- a que don Rodrigo Manrique debería dejar el mundo \comillas{sin atender} o \comillas{seguir su} \emph{falago}, un sentido más o menos similar. Quizá la lección \emph{y su }[\emph{f}-\emph{h}]\emph{alago} incurre en una \emph{lectio facilior}.};\\
v\emph{uest}ro coraçó\emph{n} de azero\\
muestre su esfuerço famoso\\ 
en este trago.\par
Y pues de vida y de salud\\
fezistes ta\emph{n} poca cuenta\\     
por la fama,\\
\edtext{fazedla de la virtud}{\Afootnote{esfuercese la virtud \emph{δ}, {\mn}, {\im}; esforçad vuestra virtud {\hh}}}\\
para sofrir \edtext{esta}{\Afootnote{aquesta {\hh}}} afrenta\\
q\emph{ue} \edtext{vos}{\Afootnote{os {\im}}} llama.\par                        
\pend

\begin{center}
	[XXXV]\footnoteA{[XXXV] XXIII en {\jm}.}
\end{center}
\pstart
»No se os faga ta\emph{n} amarga\\
la batalla temerosa\\
que esperáis,\\
pues otra vida \edtext{más}{\lemma{mas}\Afootnote{tan {\eg}}} larga\\
\edtext{de fama}{\Afootnote{dela fama {\rl}}} ta\emph{n} glorïosa\\ 
acá dexáis;\par
\edtext{que aunque esta vida de honor}{\Afootnote{aun que vida esta de honor {\im}}}\\
\edtext{tanpoco no es}{\Afootnote{no penseis ques {\mn}; tan poco no \emph{om.} {\im}}} eternal\\
\edtext{ni}{\Afootnote{\emph{om.} {\im}}} \edtext{duradera}{\Afootnote{verdadera \emph{δ}, {\mn}, {\im}}},\\
mas co\emph{n} todo es muy mejor\\
que la \edtext{otra}{\Afootnote{corona {\jm}}} \edtext{temporal}{\Afootnote{ynfernal {\eg}; corporal {\mn}}},\\     
pereçedera.\par
\pend

\begin{center}
	[XXXVI]\footnoteA{[XXXVI] XXIV en {\jm}.}
\end{center}
\pstart
»El bivir q\emph{ue} es perdurable\\
no se gana co\emph{n} estados\\ 
mu\emph{n}danales,\\
ni co\emph{n} vida deleitable\\
en que mora\emph{n} los pecados\\
infernales;\par
\edtext{que los}{\Afootnote{mas los \emph{δ}, {\mn}}} buenos religiosos\\
\edtext{gánanlo}{\lemma{gananlo}\Afootnote{ganan la {\eg}, {\jm}; ganan los {\rl}}} co\emph{n} oraçiones\\                                           
y con lloros;\\
los cavall\emph{er}os famosos\\
co\emph{n} trabajos y afliçiones\\ 
contra moros.\par
\pend

\begin{center}
	[XXXVII]
\end{center}
\pstart
»\edtext{Y}{\lemma{y}\Afootnote{\emph{om.} {\jm}}} pues vos, claro baró\emph{n},\\
tanta sangre derramastes\\                                                                    
de paganos,\\
esperad el gualardó\emph{n}\\
que en este mu\emph{n}do ganastes\\
por las manos;\par
y co\emph{n} esta confiança\\
y co\emph{n} la fe ta\emph{n} entera\\
que tenéis,\\
partid con buena\edtext{}{\Afootnote[nosep]{bua/bna/vna {\eg}}}\footnoteE{b[uen]a {\eg} (?). En el manuscrito hay una mancha de tinta borrada en el folio arriba de «ua»/«na», que no propicia distinguir correctamente si es un signo de contracción  ̅ , leyéndose así «buena»/«buena» --el copista confunde \emph{n} y \emph{u} incluso en otros lugares del manuscrito--, o constituye un error, «bua». \textcite{PérezPriego1990,PérezPriego2017} y \textcite{Severin2000} leen en conjunto «vna»; Beltrán lee «bua», y sostiene que yace «con corrección, o abreviatura poco visible, interlineada» \textcite[123]{Beltrán1991}.} espera\emph{n}ça,\\
\edtext{que la vida verdadera}{\Afootnote{estotra vida tercera \emph{δ}, {\mn}, {\im}}}\\                                                                           
\edtext{cobraréis}{\lemma{cobrareys}\Afootnote{ganareys \emph{δ}, {\mn}, {\im}}}.»\par
\pend

\begin{center}
	[XXXVIII]\footnoteA{[XXXVIII] rresponde don rrodrigo manrrique a la muerte {\eg}; Respuesta del maestre {\jm}; Reza a la muerte {\rl}; Respondel Maestre {\mn}, {\im}.}
\end{center}
\pstart
«No \edtext{gastemos}{\Afootnote{tengamos {\rl}}} tie\emph{n}po ya\\
en esta vida mezq\emph{ui}na\\ 
por tal modo,\\
que mi volu\emph{n}tad está\\
conforme con la divina\\                                             
para todo.\par
Y co\emph{n}sie\emph{n}to en mi morir\\
co\emph{n} voluntad plaze\emph{n}tera,\\  
clara y pura,\\
q\emph{ue} q\emph{ue}rer onbre bivir\\
q\emph{ua}ndo Dios q\emph{ui}ere q\emph{ue} muera,\\
\edtext{es}{\Afootnote{es grand {\mn}}} locura.\par
\pend

\begin{center}
	[XXXIX]\footnoteA{[XXXIX] oraçion que fizo el maestre {\eg}; Oracion {\jm}, {\rl}, {\im}; Oracion quel Maestre face á Dios {\mn}.}
\end{center}
\pstart
»Tú, q\emph{ue} por n\emph{uest}ra maldat\\
tomaste forma \edtext{çevil}{\lemma{çeuil}\Afootnote{seruil {\rl}, {\mn}}}\footnoteE{«çeuil» tiene la acepción de «villano», hombre de la villa, opuesto a noble, según \textcite{Lida1947}, quien expone una serie de testimonios tales como \emph{Triunfos} de Juan de Padilla, el Cartujano, y \emph{Diálogo de la lengua} de Juan de Valdés, más otros para ilustrar la historia semántica del vocablo, unido a su hipótesis de que \emph{civil} tenía una acepción adicional, «cruel», según textos de don Juan de Mena. El \emph{Tesoro de la lengua castellana} de \textcite{Covarrubias1611} registra para \emph{cevil} la definición «hombre apocado, y miserable, de ce, que acrecienta la sinificacion, y de vil, que valdra muy vil». A esto acompaña la definición para \emph{civil} «desestimable, mezquino, ruin, y de baxa condicion y procedéres» (\emph{Autoridades}, 3ª acepción), que recuerdan \textcite{CorominasYPascual1980-1991} y añaden «... actualmente ya se ha anticuado, pero de la que hay multitud de ejemplos desde la \emph{Gr. Conq. de Ultr.} hasta el s. XVII (Salas Barbadillo)». Orduna sostiene que la copla «reproduce en sus cuatro partes evidentes el mismo esquema de motivos. Los primeros 9 versos resumen la invocación a Cristo Salvador del grupo 4-6, llevándola a un plano de comunicación directa [...] los últimos 3 versos concluyen la Oración [...] con el sometimiento total a la clemencia del Salvador» \parencite[145-146]{Orduna1967}; pero esto lo sostiene --siguiendo a \textcite{Foulché-Delbosc1912}-- con la lección \emph{servil}, que \textcite{Beltrán1991,Beltrán2013} descarta por \emph{lectio facilior}.}\\
y \edtext{baxo}{\Afootnote{\emph{om.} {\jm}}} no\emph{n}bre;\\
\edtext{Tú, que a tu divinidat}{\lemma{tu que a tu}\Afootnote{juntaste {\jm}}}\\
ju\emph{n}taste \edtext{cosa}{\Afootnote{avna cosa {\jm}}} ta\emph{n} vil\\
como \edtext{el}{\Afootnote{es el {\eg}, {\rl}}} onbre;\par
Tú, \edtext{que los}{\Afootnote{que tan \emph{δ}, {\mn}, {\im}}} gra\emph{n}des torme\emph{n}tos\\
sofriste sin resiste\emph{n}çia\\
en tu persona,\\
no por mis meresçimie\emph{n}tos,\\
mas por tu sola cleme\emph{n}çia\\ 
me perdona.»\par
\pend

\begin{center}
    [XL]\footnoteA{[XL] fin {\eg}, {\rl}, {\mn}; Torna el actor y faze fin {\jm}.}
\end{center}
\pstart
Así, co\emph{n} tal ente\emph{n}der,\\
todos sentidos humanos\\
\edtext{conservados}{\lemma{conseruados}\Afootnote{oluidados {\im}}}\footnoteE{«oluidados» pertenece a {\im}, testimonio representante del subarquetipo \emph{β}. La elección entre las adiáforas es en extremo compleja porque, siguiendo el \emph{stemma codicum} de \textcite{PérezPriego2017} --y el de Beltrán, en parte--, cada una tiene igual peso en los subarquetipos \emph{α} y \emph{β} por remitir al arquetipo \emph{X}. Cada lección tuvo defensores o «electores»; para la lección \emph{conservados}, \textcite{PérezPriego1990,PérezPriego2017}, \textcite{Conde2009} y \textcite{Gonzalez2019}, y para \emph{olvidados}, su innovador y elector \textcite{Beltrán1991,Beltrán2013,Beltrán2016}. Beltrán escuetamente argumenta en favor de la lección \emph{olvidados} en su edición crítica con el siguiente argumento: «\emph{oluidados} y \emph{conseruados} se reparten en las ramas del \emph{stemma} [...] Prefiero la primera opción, pues \emph{sentido} tiene la acepción "apetito, o parte inferior del hombre" (\emph{Aut.}) reforzado probablemente por el calificativo \emph{humanos}, lo que hace de esta variante un caso evidente de \emph{lectio difficilior}.» \parencite[158]{Beltrán1991}. Tras el comentario de \textcite{Conde2009}, en su edición de 2013, Beltrán le otorga mayor autoridad al \emph{Diccionario de Autoridades} --como diccionario antiguo--, porque refleja «valores arcaicos» y conserva «interpretaciones específicas de las ciencias o el pensamiento antiguo», y porque no es una cuestión puramente léxica, sino de valoración moral --en el vocablo--; y en su extenso estudio del 2016 añade, además, ejemplos textuales de época para justificar la acepción «apetitos». [La lección \emph{conservados} forma parte de la tradición \emph{vulgata} del texto]. \textcite{Conde2009} sostiene, contra Beltrán, que del mismo modo que el editor utiliza la segunda acepción para \emph{sentidos} [apetito], podría utilizar la principal, \emph{potencia}/\emph{facultad} para percibir «las impresiones de los objetos exteriores», y la tercera, \emph{entendimiento}/\emph{razón}, por tanto, no resultaría plausible utilizar la evidencia lexicográfica para adoptar \emph{olvidados} como lección para la \emph{constitutio textus}; y proporciona, además, diferentes testimonios para justificar \emph{conservados} y, de paso, la preeminencia de la 1ª y 3ª acepción en el \emph{Diccionario de Autoridades} para \emph{sentidos}, esto es, \emph{entendimiento}/\emph{razón}. \textcite{PérezPriego2017} sigue a Conde. \textcite{Gonzalez2019} propone que la lección \emph{olvidados} resulta de una trivialización; luego, en su análisis, indica que los \emph{sentidos} no son cinco, sino diez, según la tradición medieval: cinco internos y cinco externos; los externos, los propios de la carne; los internos, siguiendo a Tomás de Aquino [\emph{S. Theol.}, I, q. 78, a. 4] --aquel, a su vez, siguiendo a Avicena y su \emph{De anima}--, son la \emph{sensus communis}, \emph{imaginatio}, \emph{phantasia}, \emph{aestimativa} y \emph{memorativa}. Luego, cita dos textos de autoridades que, con certeza, leyó Manrique: La \comillas{Partida Segunda} de Alfonso X el Sabio [\emph{Siete Partidas}, II, XIII] y la \emph{Coronación del Marqués de Santillana} de Juan de Mena [pp. 90-92 en la edición de Kerkhof]. Siguiendo esas influencias, resulta en una unidad entre \emph{sentidos} y \emph{entendimiento}, y olvidar los \emph{sentidos}, como indica la lección \emph{olvidados}, implicaría en quebrar tal unidad al \comillas{deshacerse} de ambas; y al tratarse de una unidad --siguiendo la evidencia proporcionada acerca de las \emph{ars moriendi} y la tradición--, la lección, como se estudia en ecdótica, no puede ir contra el propio autor.\par
Puede concluirse que Rodrigo Manrique fallece conservando facultades físicas y psicológicas [\emph{sentidos} y  \emph{entendimiento}], y que fallece en una «recta ley», ejerciendo los deberes propios que le corresponden como \emph{bellator}, esto es, «guerreando» \parencite[52-55]{Gonzalez2019}. La evidencia histórica y la reconstrucción del \emph{contextus} se inclinan hacia la lección \emph{conservados}.},\\
çercado de su muger\\
\edtext{}{\linenum{|||||480}\Afootnote[nosep]{\emph{sec. m.} {\eg}}}\edtext{y de sus fijos y hermanos}{\Afootnote{de sus hijos y ermanos {\eg}; y de fijos y de ermanos {\jm}, {\im}}}\\
y criados,\par
dio el alma a quien \edtext{ge la dio}{\lemma{geladio}\Afootnote{gloria dio {\eg}}},\\
el qual \edtext{la ponga}{\lemma{laponga}\Afootnote{ladio {\rl}}} en el cielo\\
y en su gloria;\\
\edtext{q\emph{ue} aunq\emph{ue}}{\Afootnote{y aunque {\jm}, {\mn}, {\im}}} la vida \edtext{murió}{\lemma{murio}\Afootnote{perdio {\eg}, {\rl}}}\footnoteE{«murió» en {\jm}, {\mn} y {\im}. Continuando el argumento anterior, se opta por la lección \emph{murió} porque, al fallecer Rodrigo Manrique con todos sus «sentidos humanos conservados», él «se conservaría» a través de la memoria de sus allegados, cosa que no acontecería si «perdiese» la vida [carnal], porque «perder» tiene connotaciones negativas según las definiciones «ocasionar algún daño a las cosas, desmejorándolas o desluciéndolas» y «padecer algún daño, ruína o disminución en lo material, inmaterial o espiritual» (\emph{Autoridades}, acepciones 6ª y 7ª). Si la «perdiese», aceptando estas acepciones, podría «dañar» o «desmejorar» los \emph{sentidos} y el \emph{entendímiento}, facultades que «regresan» en «plena forma» a su dios, como ya demostraron \textcite{Conde2009} y \textcite{Gonzalez2019}. De ser así, \emph{perdió} anularía la lección \emph{conservados} y, por tanto, lo correcto sería \emph{murió} que, además, por pertenecer a los subarquetipos \emph{α} --{\jm} y {\mn}-- y \emph{β} --{\im}--, restituye la lección del arquetipo \emph{X}.},\\
\edtext{dexónos}{\lemma{dexonos}\Afootnote{nos dexo {\jm}, {\mn}, {\im}}} harto consuelo\\ 
su memoria.\par
\pend
\endnumbering
\vfill
\relax
\newpage
%
\section*{\centering \textsc{Apéndice}}
\addcontentsline{toc}{section}{\textsc{Apéndice}}

Coplas apócrifas o póstumas en la \emph{Glosa Famossisima} de Alonso de Cervantes, ubicadas entre las coplas XXIV y XXV. Texto fijado por \textcite{Foulché-Delbosc1902}.\par\vspace{5pt}
%
\begin{center}
[I]
\end{center}
\beginnumbering
\pstart
		Es tu comienço lloroso,\\
		tu salida syempre amarga\\
		y nunca buena,\\
		lo de en medio trabajoso,\\
		a quien das vida mas larga\\
		le das pena;\par
		anse los bienes muriendo,\\
		y con sudor se procuran\\
		y los das,\\
		y los males vienen corriendo,\\
		y después de venidos\\
		duran mas.\par
		\pend
		\endnumbering
		\vspace*{11pt}
%
\begin{center}
[II]
\end{center}
		\beginnumbering
		\pstart
		O mundo, pues que nos matas,\\
		fuera la vida que diste\\
		toda vida,\\
		mas segun aca nos tratas,\\
		lo mejor y menos triste\\
		es la partida\par
		de tu vida tan cubierta\\
		de males, y de dolores\\
		tan poblada,\\
		de los bienes tan desierta,\\
		de plazeres y dulçores\\
		despoblada.\par
		\pend
		\endnumbering\par
%
\vspace{11pt} Tras la copla XL, en {\im} yace un epitafio con la rúbrica \comillas{En su sepultura dize desta manera}, que enuncia:\vspace{5pt}\par
%
\comillas{Aquí yaze muerto el ombre\\
	que vivo queda su nombre}.\footnoteE{Beltrán, citando a Luis de Salazar y Castro, recupera «AQVI YACE MVERTO VN HOMBRE, QVE VIVO DEJO SV NOMBRE» \parencite[158]{Beltrán1991}.}
\relax
\vfill
%
\end{document}